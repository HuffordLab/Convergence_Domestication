\title{Convergence of adaptation after domestication in the grasses}
\author{
        Woodhouse, M.R. and M. B. Hufford
}

\documentclass[12pt]{article}

\begin{document}
\maketitle

\begin{abstract}
The convergent evolution of desirable traits in crops during domestication has been well studied. In this review, the authors explore the current research to determine whether domestication bottlenecks in grass crops constrain environmental adaptation convergently, and how this phenomenon can help our understanding of the limitations in crop environmental adaptation.
\end{abstract}

\section{Introduction}
Most human societies across the globe rely on domesticated crop species for survival.  If crop production is a measure of consumption, in 2016 alone the United States produced 384 million tons of maize, China produced 131.6 million tons of wheat, and Nigeria produced 6.9 million tons of sorghum http://www.fao.org. Almost none of these domesticated crops are physiologically the same as they were prior to domestication; in the last 10,000 or so years, each of these crop species has been selected by humans for traits enriched for nutrition, yield, and other attractive features. As such, domesticated crops are often radically different from their wild relatives.

The process of domestication, wherein desirable traits in a wild species are selected for and bred by humans, is ultimately the process of genetic selection doi.org/10.1016/j.cell.2006.12.006  Cell. 2006 Dec 29; 127(7):1309-21. By selectively breeding individuals in, say, a maize population that have larger ears than other individuals in the population, one is also selecting for the alleles that genetically code for the creation of larger ears.  Notably, there are certain traits that distinguish domesticated crops from their wild progenitors, even among distantly related crops such as maize and sunflower; these traits include apical dominance or lack of branching, loss of seed dormancy, a loss of bitter taste in favor of sweeter flavor, larger fruits or grains, and a loss of shattering or seed dispersal (Table 1). This suite of shared traits is known as domestication syndrome. doi.org/10.1007/BF02098682 

How can two vastly diverged species such as maize and sunflower share the same domesticated traits? After all, the domestication of both maize and sunflower happened millions of years after these two species diverged. Since these two species still share some enzymatic pathways, perhaps similar genes within these pathways have alleles that code for similar traits advantageous to domestication. Shared phenotypes that are due to the modification of similar genes, or orthologs, is a phenomenon sometimes known as parallelism; parallelism is more likely to occur the more closely related any two species are.   https://www.frontiersin.org/articles/10.3389/fevo.2018.00056/full. Conversely, it is possible for two different genes in two different enzymatic pathways in two different species to give rise to alleles which happen to code for similar traits, especially if both species experience similar selection pressures (either human or environmental). This phenomenon is known as convergence, and in diverged species it is more likely than parallelism to result in similar phenotypic traits, since there are fewer shared pathways between distantly related species.

After the initial wave of crop domestication yielded the aforementioned domestication syndrome traits, another level of domestication has since ensued, which is the adaptation of crop species to specific environments.  At both the global and the local level, distribution of crops to new environments has made it necessary for breeders to select traits that are conducive to the new environment in question. A cultivar of maize bred to thrive at sea level, for instance, may not necessarily thrive in the colder, UV-intensive environment of the Andes.  Therefore, a breeder in the Andes must now look for individuals in the domesticated maize population that are hardy under the new conditions. However, crop adaptation faces genetic limitations that the domestication of wild progenitors never had, and that is due to the genetic bottleneck. 

If domestication leads to the genetic selection of certain alleles that code for certain desirable traits (such as those in domestication syndrome), then other alleles with either neutral or undesirable traits are going to be lost during the selection process, particularly if the breeding population is small. The diversity of a population is, in general, proportional to the number of genetically distinct individuals within that population. Loss of genetic diversity in a population is defined as a genetic bottleneck. A population with no genetic diversity at all is a monoculture. It was a monoculture of the “lumper” variety of potato that led to the infamous Potato Famine in Ireland in the 1840s.  The Potato Famine demonstrated that by divesting a crop cultivar of its diversity, the cultivar also loses its ability to adapt to environmental pressures, because the alleles that code for adaptive traits such as, for instance, disease resistance are lost.  

The ways in which domestication bottlenecks have shaped the potential for adaptation in crops is the topic of this review. We will focus mainly on grass crops, since the major grass crop species—maize, rice, sorghum, wheat, and barley—are diverged enough to where an analysis of trait convergence is feasible, but closely related enough to share enzymatic pathways useful in studying parallelism. In addition, grasses share a certain amount of genomic dynamism, including polyploidization and transposable element activity, that can give rise to new alleles which can contribute to both domestication and adaptation.  In this review we will discuss to what extent the same alleles that resulted in domestication syndrome in grass crops also play a potential role in adaptation, which adaptive traits are expected to be convergent or parallel, and if it is possible to predict with any certainty which loci are likely to yield adaptive alleles. \paragraph{}

\section{Grasses as a single genetic system to study domestication}
It has been theorized that the grasses are closely related enough, yet divergent enough, to be viewed as a single genetic system (Bennetzen and Freeling 1993, Freeling 2001). This makes grasses especially conducive to a meaningful study of convergence. Grasses share certain features that permit similar types of selection, particularly dynamic genomes. All grasses share a genome duplication event 70-90 MYA 10.1073/pnas.0307901101, with maize undergoing a later ancient polyploidy event, and wheat having even more recent polyploids, including the tetraploid Triticum turgidum and the hexaploid T. aestivum, or modern bread wheat (refs needed). Grasses tend to  have relatively large genomes with active transposons, particularly maize and wheat (refs); they share certain domestication phenotypes, such as shatterproof seed, larger seed size, and early flowering (refs); and most grass crops were domesticated within the latitudinal boundaries of the equator and 35 N (Hawkes 1983; Harlan 1992; Smartt and Simmonds 1995), under similar environmental conditions, featuring both wet and dry seasons (Harlan 1992).  Nevertheless, each grass cereal has been cultivated separately in separate geographic locations, including maize (Americas), Sorghum (Africa), Rice (Asia), and Wheat and Barley (Middle East) (Note: not sure how relevant the last two points are, since I don't explore them at all later. However, if you have stuff to add that is germane to these topics, by all means, add to it! -MW). \paragraph{}

\section{Domestication bottlenecks in grass crops}
Domestication bottlenecks are often a result of selection for traits that make for desirable crops but not necessarily for environmental adaptability. Massive nucleotide diversity loss is reported in domesticated bread wheat (doi:10.1093/molbev/msm077), maize (with an increase in deleterious alleles) (PMID: 9539756, DOI 10.1186/s13059-017-1346-4), rice (PMID: 17218640), Sorghum (doi:  10.1534/genetics.105.054312), and barley (https://doi.org/10.1007/s00438-006-0136-6) compared with wild relatives, demonstrating that loss of diversity is widespread in cultivated grasses and is a phenomenon that is distinct from uncultivated wild relatives. Together these results suggest that domestication itself is responsible for the loss of diversity, and because of this, attempts to adapt domesticated grasses to new environments could pose a challenge. (This section might be a little thin. --MW)

\section{Post-domestication adaptation in grass crops: likelihood of convergence}
To what extent does diversity loss impact crop adaptation? Given how dynamic the grass genomes tend to be, as discussed, we would expect that even with a reduction in diversity, cereal cultivars could still adapt to environmental changes faster than less genomically dynamic species. The question is whether the initial loss of diversity always impacts the same genes that confer adaptation within these various cereal crops. Some lines of evidence would suggest this is not the case. Orphan genes, or genes that are specific to a particular lineage and share no syntenic orthologs with any outgroup, are known to code for traits involved in plant defense and stress response (http://dx.doi.org/10.1105/tpc.111; reviewed in https://doi.org/10.1016/j.tplants.2014.07.003). Orphan genes tend to be very dynamic, arising and becoming lost much faster than their syntenically retained, basal counterparts (http://www.genome.org/cgi/doi/10.1101/gr.081026.108). They do not tend to be involved in essential cellular or housekeeping processes, nor usually in development, which means they could be easily lost during domestication. If adaptation is dependent on these orphan-type genes, which quite often are unique even in individual cultivars within the same crop species, then we would not expect to see convergence at the alleleic level in cereal adaptation, since each species--indeed, each cultivar--would be expected to have its own unique, "outward-facing" suite of orphan genes that would confer environmental adaptation uniquely to its niche. Orphan genes often propagate through trans duplication (http://www.genome.org/cgi/doi/10.1101/gr.081026.108., https://doi.org/10.1371/journal.pgen.1000949); therefore, movement of these genes to a new region whose local euchromatic status can confer novel expression patterns to the mobilized gene can be a strong source of adaptation, especially since it has been shown that stressful environments can stimulate activation of transposable elements (https://doi.org/10.1104/pp.127.1.212; https://doi.org/10.1371/journal.pgen.1004915; reviewed in 10.3389/fpls.2016.01448), and perhaps the same or similar mechanisms can give rise to the mobilization of orphan genes. Taken together, these data suggest that environmental adaptation among different cereals would not be convergent, at least in regards to defense or stress response.

On the other hand, there are other traits thought to be involved in adaptation (?????Is what I say about these genes true, do you think? if you have better examples, that would be great! and we probably need some real-world examples of this kind of adaptation if it exists!-MW) that are often syntenically retained throughout the grasses, such as color genes (r/b, 10.1534/genetics.104.034629), vernalization (Vrn1, PMID: 12454082) and anoxic genes (Adh1/2, Clegg et al 199).  Some of these genes were likely originally part of the domestication process themselves, and so it stands to reason that they could possibly revert to an older, more adaptive phenotype if mutant alleles reverting it were selected for in a new environment favorable to it, which might happen more quickly in the grasses than in other types of crops; after all, the same genomic dynamism that can quickly evolve orphan genes can of course cause activated TEs themselves to transpose near genes that might confer novel expression patterns that are conducive to adaptation. The chances of a basal, syntenically retained domestication gene reverting to a more adaptive allele has therefore a higher likelihood to be convergent. However, if such "retro-adaptive" domestication genes existed as a single copy, then reversion to an adaptive phenotype could mean that the traits that made them desirable for domestication in the first place would be lost, and thus such alleles may not be selected for. But if such genes had post-polyploid homeologs, then their retained counterpart might become mutated instead so as to confer fitness without losing the desired domestication phenotype (Note: this is very hand-wave-y, I realize. But this is why we are writing this together, so you can tell me if I am full of it! :D -MW). This phenomenon of course would be reserved only for genomes such as maize or wheat that have had a more recent polyploid event. To some extent, however, it can be predicted which homeolog would be adaptive. It is known that of the two retained post-polyploidy subgenomes in maize, one undergoes less fractionation and is more highly expressed than the other (i.e. the dominant subgenome) (https://doi.org/10.1371/journal.pbio.1000409, https://doi.org/10.1073/pnas.1101368108). There is evidence that fractionation is biased not only in maize, but in wheat as well (10.1105/tpc.114.127183). Recent work has suggested that the genes on the more highly expressed subgenome in maize contribute more to phenotypic variation (doi:10.1093/molbev/msx121), and so it would be expected that selected adaptive alleles would likely be on the retained duplicate in the more dominant subgenome in both maize and wheat.  On the other hand, it is possible that subfunctionalization of both retained duplicates has already occurred, and that if the homeolog on the dominant subgenome is already conferring the domestication trait, plants with mutations on the homeolog on the nondominant subgenome might be more selected for adaptation. (Not really sure if genome dominance really adds much to the convergence argument, especially since the argument I use is very inconclusive. --MW)

\section{Conclusions}
The extent to which adaptation of cereal crops can be convergent is dependent on the environment the cultivar will be adapting to, since different adaptive phenotypes are dependent on either fast-evolving genes (stress- or defense response) or more basal genes (color, vernalization, other retro-adaptive domestication traits). Therefore, adaptation of cereal crops can be to some extent predictive, depending on the environment or the likelihood of adaptive alleles arising within certain loci within a population. 

\paragraph{Outline}
 Section~\ref{results}.
Finally, Section~\ref{conclusions} 



\bibliographystyle{abbrv}
\bibliography{simple}

\end{document}
This is never printed
