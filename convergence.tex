\documentclass[12pt]{article}

\title{The limits of parallelism in adaptation due to domestication in the grasses}
\author{Woodhouse, M.R. and M. B. Hufford}

\usepackage[utf8]{inputenc}
\usepackage{colortbl}
\usepackage[letterpaper, margin=1in]{geometry} %package that allows changes in margins and header/footers
\usepackage{authblk}%allows footnote format for authors
\usepackage{rotating}
\usepackage{amsmath}
\usepackage{array}
\usepackage{booktabs}
\usepackage[x11names,dvipsnames,table]{xcolor}
\usepackage{tabularx}
\usepackage{mathptmx}       % selects Times Roman as basic font
\usepackage{helvet}         % selects Helvetica as sans-serif font
\usepackage{courier}        % selects Courier as typewriter font
\usepackage{type1cm}        % activate if the above 3 fonts are
                            % not available on your system
%
\usepackage{natbib}
\usepackage{makeidx}         % allows index generation
\usepackage{graphicx}        % standard LaTeX graphics tool
                             % when including figure files
\usepackage{multicol}        % used for the two-column index
\usepackage[bottom]{footmisc}% places footnotes at page bottom
\usepackage{setspace}
\usepackage{gensymb}
\usepackage{color}
\usepackage[textsize=tiny,colorinlistoftodos]{todonotes} % comments in margins

\definecolor{cornflowerblue}{rgb}{0.39, 0.58, 0.93}
\newcolumntype{G}[1]{>{\raggedright\let\newline\\\arraybackslash\hspace{0pt}}m{#1}}
\newcolumntype{C}[1]{>{\centering\let\newline\\\arraybackslash\hspace{0pt}}m{#1}}
\newcommand{\mbh}[1]{\textcolor{red}{\normalsize  #1}}
\newcommand{\mw}[1]{\textcolor{LimeGreen}{\normalsize #1}}

\begin{document}
\maketitle

\begin{abstract}
The selection of desirable traits in crops during domestication has been well studied.
Many crops share a suite of modified phenotypic characteristics collectively known as the domestication syndrome.
In this sense, crops have convergently evolved.
Previous work has demonstrated that, at least in some instances, convergence for domestication traits has been achieved through parallel molecular means.
However, both demography and selection during domestication may have placed limits on evolutionary potential and reduced opportunities for convergent adaptation during post-domestication migration to new environments.
Here we review current knowledge regarding trait convergence in the cereal grasses and consider whether the complexity and dynamism of cereal genomes (e.g., transposable elements, polyploidy, genome size) helped these species overcome potential limitations due to domestication and achieve broad subsequent adaptation.
\end{abstract}

\section*{Introduction}
Certain species of plants have been continually selected over the last 10,000 years to better meet the needs of humans.
Similar selection pressure has favored traits that consistently distinguish these domesticated crops from their wild progenitors \citep{Doebley2006}, distinctions that are shared even among distantly related species such as maize and sunflower.
These traits include increased yield, apical dominance or lack of branching, loss of seed dormancy, loss of bitterness, and loss of shattering or seed dispersal.
Collectively, this suite of shared traits is known as the domestication syndrome \citep{Hammer1984}.

Trait sharing among diverged species such as maize and sunflower (their last common ancestor was 150 MYA \citep{Chang2004}) is an example of the phenomenon of convergence. 
Convergent traits can arise from unrelated genes in different enzymatic pathways, such as fruit/seed indehiscence in both dicot and monocot crops (reviewed in \citep{Dong2015}).
But if a convergent trait is caused by repeated modification of the \emph{same} molecular pathway, ortholog, or nucleotide, we define this as parallelism \citep{Rosenblum2014} (Figure \ref{fig:convergence}). 
We expect that parallelism is more likely to occur in closely related species due to their similar complement of genes and pathways \citep{Pickersgill2018}, and less likely to occur in substantially diverged species that contain fewer orthologous loci and pathways \citep{Washburn2016, Pickersgill2018}.

After the initial wave of crop domestication yielded convergence in many of the aforementioned domestication syndrome traits, another period of trait evolution ensued--the adaptation of crop species to varied environmental conditions and pathogens during global expansion.
A variety of maize bred for cultivation at sea level, for instance, may not necessarily thrive in the colder, higher UV environment of the Andes mountain range.
Therefore, cultivators in the Andes must have looked for individuals in the existing domesticated maize population that were hardy under these new conditions.
However, crop adaptation occurred under genetic limitations not experienced during domestication of wild progenitors \citep{Wang2017}. 

Only a subset of genome-wide diversity was retained in initial domesticates and additional diversity was lost through subsampling events during crop expansion.
Furthermore, selection on particular alleles coding for domestication traits often resulted in dramatic reductions in diversity in particular chromosomal regions.
The effects of this loss of genetic diversity on the potential for adaptation has been documented.
For example, a dramatic genetic bottleneck in the ``lumper" variety of potato led to a catastrophic outbreak of \emph{Phytopthera infestans}, resulting in the infamous Potato Famine in Ireland in the 1840s \citep{Goodwin1994}.
The Potato Famine demonstrated that, by divesting a crop cultivar of its diversity, the cultivar loses its ability to adapt to newly encountered environmental pressures, because the alleles that code for adaptive traits such as, for instance, disease resistance are lost.

This review will consider the extent to which selection and bottlenecks during domestication have affected the potential for parallel adaptation post-domestication.
We will focus mainly on cereal grass crops since the major domesticates--maize, rice, sorghum, wheat, barley, and millet--include a range of divergence times conducive to both parallel domestication and adaptation.
While the cereals have documented loss of diversity due to domestication and subsequent expansion, they possess dynamic genomes with frequent polyploidization, transposable element (TE) activity, and labile genome size.
These features may have provided cereals with an advantage in escaping the limits of domestication by generating novel diversity upon which adaptation could act.
\mbh{intro. is o.k.}
\mbh{be sure to include summary in conclusion about the agronomic relevance of studying parallelism}
\paragraph{}

\section*{The effects of domestication on adaptation in the cereals}

\subsection*{Domestication in the cereals}
Cereal grasses have often been studied as a cohesive genetic group \citep{pmid8379002, pmid11244100}, and there are many reasons why they present a compelling system for studying crop domestication and adaptation.
The grass clade is thought to have arisen around 75 MYA \citep{BOUCHENAKKHELLADI2010, Kellogg2001}, eventually leading to the rice, wheat, barley, millet, maize, and sorghum lineages (Figure \ref{fig:grassphylo}).
Prior to the radiation of the grasses, however, a genome duplication event occurred approximately 70 MYA \citep{Paterson2004}, which is shared among all grass crops (Figure \ref{fig:grassphylo}).
Subsequently, both maize and wheat have undergone additional, lineage-specific polyploidy events (Figure \ref{fig:grassphylo}) \citep{Levy2002}.
These polyploidy events, followed by selective and ongoing fractionation, present an opportunity for grass genomes to evolve subfunctionalized homeologs; this, along with relatively high transposon activity (particularly in maize and wheat) \citep{Wicker2016, Lisch2001}, provides substantial functional diversity upon which selection can act during domestication and adaptation.

Collectively, cereals have been targeted by human selection for millable grain.
Components of the domestication syndrome convergently selected within grain include increased seed size, loss of seed dormancy, loss of bitterness, loss of shattering or seed dispersal \citep{Lenser2013}, fragrance \citep{Kellogg2001}, and glutinous seeds \citep{Meyer2013}.
Plant architecture traits such as apical dominance or lack of branching have also been convergently targeted \citep{Lenser2013}.
A number of well characterized or candidate domestication genes are known in the cereals and have been described in Table \ref{tab:Ortho}.
This table includes an expanded set of loci and information beyond that originally published in \citep{Lenser2013}. 
While parallelism can occur for convergent traits at the level of the nucleotide, gene, or pathway  (Figure \ref{fig:convergence}), orthology, or shared functional genes across species, is currently best characterized and is therefore our primary focus.
Cereal domestication genes are categorized based on whether they occur strictly within a species, share orthologs across the grasses, share orthologs within and outside of the grasses, or share orthologs entirely outside of the grasses (Column 5, Table \ref{tab:Ortho}). 
This provides an opportunity to evaluate the extent of parallelism for a convergent domestication trait. 
In Column 1 of Table \ref{tab:Ortho}, we indicate whether a gene is thought to be associated with a domestication trait, and in Column 8 whether the gene is expected to be selected in parallel, depending on whether it has known orthologs in other species that are associated with the given domestication trait. 
For instance, fragrance is a convergent domestication trait that is shared among species as diverse as rice and soybean, and it is likely to have been selected in parallel, since orthologs of the fragrance-associated gene BADH2 have been independently targeted (Table \ref{tab:Ortho}).  
Another convergent trait likely to have been selected in parallel is glutinous seeds, since the \emph{Waxy} gene that confers the glutinous trait has targeted orthologs in nearly all cereals and beyond (reviewed in \citep{Meyer2013}).
Shattering also shows evidence of parallel selection in the cereals with orthologs of \emph{Sh1} selected in sorghum, rice, and maize \citep{Lin2012}.
With regard to convergence and parallelism during domestication, one clear message emerges from evaluation of Table \ref{tab:Ortho}: while we have much left to characterize regarding the genetic basis of domestication across the cereals, both phenomena are not rare.

However demography and selection during domestication could meaningfully affect the frequency of convergence and parallelism during post-domestication adaptation.
Genome-wide loss of diversity during genetic bottlenecks associated with both initial domestication and later crop expansion may constrain adaptation by reducing the diversity in cereal crop populations.
Substantial genome-wide loss of nucleotide diversity during domestication is reported in domesticated bread wheat ~\citep{Haudry2007}, maize (with an increase in deleterious alleles) ~\citep{pmid9539756, Wang2017}, rice ~\citep{pmid17218640}, Sorghum ~\citep{Hamblin2006}, and barley ~\citep{Kilian2006} compared with wild relatives, demonstrating that loss of diversity is widespread in cultivated grasses and is a phenomenon that is distinct from uncultivated wild relatives.
Further reductions in nucleotide diversity are found near loci underlying domestication syndrome traits.
It would seem that domestication itself is responsible for the loss of diversity, and, because of this, attempts to adapt domesticated grasses to new environments could be constrained by recent demography.  
Both demographic bottlenecks and selection during domestication could affect the likelihood that adaptive traits were selected in parallel, since these depend on whether adaptive alleles are retained across taxa post-domestication.

\subsection*{Post-domestication adaptation in the cereals}

An adaptive trait is one that interacts or responds to the environment in a way that helps an organism to thrive. 
By this definition, adaptive traits can include (but are not limited to) flowering time, drought tolerance, cold tolerance, soil salinity, and pathogen defense. 
For domesticated crops, however, adaptive traits that reverse desired domestication phenotypes such as yield, fragrance, flavor, or reduced shattering would not be considered favorable; therefore, we will narrow the definition of an adaptive trait to one that interacts or responds to the environment favorably but does not detract from desired domestication traits. 

Perhaps it is also necessary to define what specifically is meant by ``environment".
A straightforward (and admittedly simplistic) way would be to break ``environment" down to discrete features, which can include, for example, the level of carbon dioxide in the air, the level of UV radiation, temperature, day length, humidity, rainfall, wind, soil nutrient load, and soil salinity. 
By dividing the environment into these discrete elements, we can address each element individually by asking what sort of adaptive trait we would expect to observe in response to each, how many of these adaptive traits are expressed in the same genetic pathways as known domestication genes, and which represent entirely distinct physiological processes.
Most cereal crops were domesticated within the latitudinal boundaries of the equator and 35 N ~\citep{Jain1993, Gepts2010}, featuring both wet and dry seasons ~\citep{Jain1993}, which means, in relation to some aspects of environment, they likely shared similar initial adaptation.
Subsequent to domestication, cereals expanded to broad distributions, encountering new pathogens, cooler temperatures, distinct photoperiod, altered growing seasons, and varied elevation.
While much attention has been given to whether convergence in domestication traits occurred through parallel molecular means, far less is known about these processes during crop expansion and adaptation to novel habitats.

\mw{To what extent are known adaptive traits convergent? One example is cold tolerance. Cold tolerance has been reported in a number of plant species, including wheat, barley, and Arabidopsis (Table \ref{tab:Ortho}). As this trait is associated with several genes orthologous across wheat and barley as well as Arabidopsis (e.g.  \textit{Wcs19} ~\citep{pmid8219063}, \textit{Wcor14} ~\citep{pmid10846621} and \textit{Bcor14b} ~\citep{pmid9952464}, \textit{COR15a}  ~\citep{pmid9826741, Takumi2003}, Table \ref{tab:Ortho}), this trait is likely to have been selected in parallel.  Another adaptive trait that is convergent is drought tolerance, a trait also observed across plant taxa (Table \ref{tab:Ortho}). The maize ZmVPP1 gene has an upstream insertion that is linked to the drought tolerance phenotype [99]. Since this gene has an ortholog also linked to drought tolerance in Arabidopsis, AVP1 [29], this convergent trait is likely to have been selected in parallel. }
\mbh{still need to work in the previous paragraph}


\paragraph{}

\section*{Predicting the likelihood of parallelism in post-domestication cereal adaptation}
The likelihood that a convergent adaptive trait will be selected in parallel is summarized in Figure \ref{fig:adaptation}.
Here, we will focus primarily on our expectations for post-domestication parallel selection of adaptive traits in the cereals relative to (1) whether an adaptive gene is orthologous in other species; (2) whether an adaptive gene functions also as a domestication gene; (3) mutation rate; and (4) mutational target size; and, finally, how the dynamics of cereal crop genomes may uniquely influence some of these expectations. 

\subsection*{Adaptive genes that have orthologs in other species}
A relatively simple way to predict the possibility of parallel adaptation is whether a gene characterized as functioning in an adaptive capacity is orthologous across taxa. 
An ever-increasing number of causal genes for adaptation are being identified both within and outside of the grasses; some of these genes are listed in Table \ref{tab:Ortho}.
In this table, adaptation genes that have been characterized can be categorized based on whether they occur strictly within a species, share orthologs across the grasses, share orthologs within and outside of the grasses, or share orthologs entirely outside of the grasses (Column 5).
This gives us an opportunity to form hypotheses regarding the likelihood of parallelism for a certain trait based on its known orthology across taxa (Column 8, Table \ref{tab:Ortho}).
If an adaptive gene is orthologous across taxa, there is at least the possibility of parallel adaptation. 

One example of a gene thought to confer adaptation is \textit{ZmCCT9} in maize, which appears to be involved in flowering under the long days of higher latitudes.
More specifically, a transposon insertion upstream of \textit{ZmCCT9} in domesticated maize cultivars led to reduced photoperiod sensitivity, which has allowed domesticated maize to expand its range ~\citep{Huang2017}. 
This gene has orthologs in barley and wheat (Column 3, Table \ref{tab:Ortho}); therefore, there is the potential that the ortholog exists in other cereal crops as well (where it has yet to be characterized).
At least, under the criterion of shared orthology, parallel convergence of this trait is possible in the cereals.
An upstream insertion near a second adaptive gene in maize, \textit{ZmVPP1}, has been linked to drought tolerance ~\citep{Wang2016}.
Since this gene has an ortholog also linked to drought tolerance in Arabidopsis, \textit{AVP1} ~\citep{Gaxiola2001}, orthologs may exist elsewhere in the cereals.
Likewise, wheat and barley share a small family of cold-tolerance genes including \textit{Wcs19} ~\citep{pmid8219063}, \textit{Wcor14} ~\citep{pmid10846621} and \textit{Bcor14b} ~\citep{pmid9952464}, all of which encode chloroplast-targeted COR proteins analogous to the Arabidopsis protein \textit{COR15a}  ~\citep{pmid9826741, Takumi2003}.
The LEA protein orthologs \textit{HVA1} and \textit{Wrab 18/19} in barley and wheat, respectively, are also associated with cold tolerance ~\citep{Hong1988, pmid16755132}.
Transcript and protein levels of the barley \textit{HvPIP2} aquaporin gene were found to be down-regulated in roots but up-regulated in the shoots of plants under salt stress ~\citep{Katsuhara2002}.
\textit{HvPIP2} has an ortholog in both maize, \textit{ZmPIP2-4} ~\citep{Zhu2005}, and spinach, \textit{PM28A} ~\citep{Fotiadis2000}.
There are also the ASR (abscisic acid, stress, and ripening-induced) genes that are associated with salinity tolerance in rice ~\citep{Joo2013}, \textit{Setaria} (millet) ~\citep{Li2017}, and tomato ~\citep{Konrad2008}.  

However, another drought-tolerance gene in rice, \textit{OsAHL1} ~\citep{Zhou2016}, does not appear to have a characterized drought-tolerant ortholog in any other species at the time of this writing.
While this ortholog may certainly exist undiscovered in other grasses, there are some adaptive traits, for example, those involved in pathogen resistance, that are less likely to have orthologs, even in closely related species.
There are examples of shared orthologs for pathogen defense and stress response genes in the grasses (Table \ref{tab:Ortho}), but by and large, genes that code for traits involved in plant defense and stress response are frequently orphan genes, or genes that are specific to a particular lineage, sharing no defined orthologs with any outgroup ~\citep{Woodhouse2011}; reviewed in ~\citep{Arendsee2014}.
Orphan genes tend to be very dynamic, arising and becoming lost much faster than their basal counterparts ~\citep{Freeling2008}.
Orphan genes can arise via transposon exaptation ~\citep{Donoghue2011} and propagate through trans duplication ~\citep{Freeling2008, Arendsee2014}, including retrotransposition ~\citep{Wang2006}. 
Therefore, movement of these genes to a new region whose local euchromatic status can confer novel expression patterns to the mobilized gene can be a strong source of adaptation, especially since it has been shown that stressful environments can stimulate activation of transposable elements ~\citep{Beguiristain2001, Makarevitch2015}  reviewed in ~\citep{Negi2016}. 
This is one way that cereal crops might be able to escape their legacy of reduced diversity due to domestication in order to adapt. 
If a convergent adaptive trait such as pathogen resistance is dependent on these orphan-type genes, which quite often are unique even in individual cultivars within the same crop species, then we would not expect to see parallel selection of this trait at the allelic level in cereal adaptation, since each species--indeed, each cultivar--would be expected to have its own unique, "outward-facing" suite of orphan genes that would confer environmental adaptation uniquely to its niche.

\subsection*{Domestication genes that have adaptive components}

A gene involved in domestication may be less likely to be selected during adaptation, if the adaptive function reverses the domestication phenotype. 
Therefore, it is useful to define genes that are known to function as domestication genes versus those that have been characterized as adaptive (Column 1, Table \ref{tab:Ortho}).
There are a number of adaptive traits described previously, such as drought tolerance, cold tolerance, soil salinity, and pathogen defense, that are unlikely to have a domestication component, since they appear unrelated to previously described domestication syndrome traits.
Some of these adaptive traits have characterized genes described in Table \ref{tab:Ortho}.
However, there are genes described in Table \ref{tab:Ortho} that can be potentially associated with both domestication and adaptation.
For example, the \emph{Ghd7} gene in rice has been associated with domestication traits such as grains per panicle; however, natural variants with reduced function allow rice to be cultivated in cooler regions ~\citep{Xue2008}, which is an adaptive phenotype. 
Another example of a domestication trait with an adaptive component is pigmentation.
Loss of pigmentation has been favored in a variety of cereal cultivars as a cultural preference during domestication.
Yet pigment assists with UV tolerance in cereals and other plant species, particularly at high elevation ~\citep{pmid8058838, Gould2004}.
Therefore, pigmentation could lead to greater tolerance of UV radiation in cereals colonizing high elevation post-domestication ~\citep{Pyhjrvi2013}.
Genes targeted during domestication do not appear to be entirely unassociated with adaptation, suggesting either standing adaptive variation has survived the domestication bottleneck and selection or mutational processes have introduced adaptive alleles post-domestication.

\subsection*{Mutation rate}
The limitation imposed by the domestication in cereal crops may, to some extent, be reversed due to their dynamic genomes, particularly their high transposon activity relative to other crop species. 
We have seen that grasses tend to have relatively active transposons, and this transposon activity may permit a higher mutation rate in cereals ~\citep{Wicker2016}. 
In Table \ref{tab:Ortho}, several adaptive phenotypes are due to a transposon insertion somewhere in the functional region of a gene, such as \textit{ZmCCT9} and \textit{ZmVVP1} in maize.
However, a comprehensive review of TEs and plant evolution ~\citep{Lisch2013} suggests that our understanding of the role of transposable element activity in crop adaptation is largely anecdotal and might be overstated, but perhaps can be better elucidated by harnessing the recent advances in genomics such as more sophisticated TE annotation protocols, whole-genome sequencing, and comparative algorithms.
Using these advances in genome biology, a recent study by Lai and coworkers found that transposon insertions may have played an important role in creating the variation in gene regulation that enabled the rapid adaptation of domesticated maize to diverse environments ~\citep{Lai2017}.

\subsection*{Mutational target size}
Transposable elements may also result in meaningful differences in genome size across cereal crops.
Transposons are known to contribute to the expansion of genome size in maize and other plant species ~\citep{Tenaillon2011} (reviewed in ~\citep{Lisch2013}).
A recent review ~\citep{Mei2018} suggests that larger genomes may affect the process of adaptation by increasing the number and location of potentially functional mutations, thus expanding the regulatory space in which functional mutations may arise.
This may increase the likelihood that a given orthologous gene or pathway could be selected in parallel for adaptive traits, despite losses of diversity experienced during domestication.
In this way, cereal grasses may be more poised than other crop species to reverse the effects of their domestication bottlenecks, and increase the chances of parallel adaptation, due to their high rates of transposon activity. 

 
Whole-genome duplication events can result in homeologs that may undergo subfunctionalization or neofunctionalization and give rise to adaptive loci.
Neofunctionalization of homeologs is widespread in maize ~\citep{Hughes2014} and in bread wheat ~\citep{Pfeifer2014}, which have both undergone recent, lineage-specfic polyploidy events (Figure \ref{fig:grassphylo}).
To some extent, it can be predicted which homeolog in a post-polyploid cereal is likely to be adaptive.
It is known that of the two retained post-polyploidy subgenomes in maize, one undergoes less fractionation and is more highly expressed than the other (i.e. the dominant subgenome) ~\citep{Woodhouse2010, Schnable2011}, and there is evidence that fractionation is biased not only in maize, but in wheat as well ~\citep{Eckardt2014}.
Schnable and Freeling found that of the characterized genes that have a known mutant phenotype, the majority are on the less fractionated subgenome ~\citep{Schnable20112}.
Many of these genes, such as \textit{tb1}, \textit{Waxy}, \textit{Opaque2}, and several starch synthesis and coloration genes not in Table \ref{tab:Ortho}, have a domestication syndrome phenotype in maize.
Additionally, recent work has suggested that the genes on the more highly expressed subgenome in maize contribute more to phenotypic variation than the less expressed subgenome across a wide variety of traits, including those linked to adaptation ~\citep{RennyByfield2017}.
Indeed, two genes associated with adaptive phenotypes in maize from Table \ref{tab:Ortho}, \textit{ZmVPP1} (drought tolerance) and \textit{ZmPIP2-4} (soil salinity) are both found on the less fractionated subgenome ~\citep{Schnable20112}; and genes associated with adaptation phenotypes such as disease resistance have also been observed on the more dominant subgenome in ~\citep{RennyByfield2017}. 
Parallelism in recent polyploids, therefore, may be more likely within the more dominant subgenome and may be aided by neofunctionalization of these homeologs.

\section*{Conclusions}

This review set out to explore the extent of convergence and molecular parallelism in domestication syndrome traits in cereals.
We have also considered how domestication has influenced the potential for subsequent parallel adaptation in the grasses during expansion to novel habitats. 
Demographic bottlenecks and targeted selection during domestication have removed potentially adaptive variation which may, in turn, reduce the extent of parallelism observed during adaptation.
We hypothesized that parallelism in adaptation in cereal crops is affected by:  (1) whether an adaptive gene is orthologous across taxa; (2) whether an adaptive trait is related to the domestication syndrome; (3) the mutation rate in cereals, particularly as influenced by transposable elements; and (4) the mutation target size.
As demonstrated in Table \ref{tab:Ortho}, the causal loci underlying adaptation in the grasses are just beginning to be discovered.
Comparative genomic analyses of cereals and their wild relatives combined with comparative studies of uniquely adapted populations will help identify genes involved in these processes and further characterize whether selection occurred in parallel.
The genomic study of time-stratified archaeological samples may also help clarify the timing of selection and the extent to which parallelism in adaptation is conditioned on initial selection during domestication.
Crops in the grass family have been very successful in adapting to a wide range of environmental conditions despite limitations in adaptive potential due to domestication, perhaps due to the their active TEs, their history of polyploidy, and their large genomes.
In addition to enhancing our basic understanding of the repeatability of evolution, this is useful if we wish to breed wild grass relatives for adaptive traits, or create hybrids among existing cultivars, since we can now associate favorable phenotypes and QTL with orthologs across species by simple comparative genomics.

\section*{Authors' Contributions}
M. Hufford devised the theme and general concepts of domestication, adaptation, and parallelism. M. Woodhouse devised the sections relating to grass genome plasticity, orthology, and orphan genes. M. Woodhouse and M. Hufford contributed equally to the drafting and the editing of the article.

\section*{Competing Interests}
The authors report no competing interests.

\section*{Funding}
This work was funded by USDA grant USDA-ARS 58-5030-7-072.

\bibliographystyle{plain}
\bibliography{convergence.bib}

\begin{sidewaystable}
\rowcolors{2}{gray!25}{white}
\begin{center}
\caption{Parallel or Convergent Orthologies (adapted from Lenser and Theissen, 2013 \citep{Lenser2013})} \label{tab:Ortho}
\fontsize{7}{8}\selectfont
\begin{tabular}{C{1.5cm}G{2.5cm}C{3.10cm}C{3.2cm}C{2.3cm}G{3.5cm}G{2.1cm}C{1.3cm}G{1.5cm}}
\\\toprule
{\bf Trait type} & {\bf Phenotypic trait} &{\bf Crop species} & {\bf Ortholog phylogeny} & {\bf Phylogeny of domestication trait} & {\bf Orthologous gene(s)} & {\bf Gene product} & {\bf Potentially parallel?} & {\bf References} \\\toprule
 domestication & Determinate growth &Tomato, soybean, common bean & Family/above family & outside the cereals & SP, Dt1, PvTFL1y & Signaling protein & parallel & \citep{Doebley2006, Repinski2012, Liu2010, Kwak2012, Tian2010}\\
 domestication & Dwarfism &Rice, barley & Family & grass-wide & OsGA20ox-2, HvGA20ox-2 & Metabolic enzyme & parallel & \citep{Asano2007, Asano2011, Jia2009}\\
 domestication & Dwarfism &Sorghum, pearl millet & Family & grass-wide & dw3, d2 & Transporter protein & parallel & \citep{Multani2003,Parvathaneni2013}\\
 domestication & Dwarfism &Wheat & Species & species-specific, grass & Rht-1 & SH2-TF & unknown & \citep{Doebley2006}\\
 domestication & Fragrance &Rice, soybean & Species/family & cereals and beyond & BADH2, GmBADH2 & Metabolic enzyme & parallel & \citep{Kovach2009, Juwattanasomran2010}\\
 domestication & Glutinous seeds &Rice, wheat, maize, foxtail millet, barley, amaranth, sorghum, broomMaize millet & Species/family/above family & cereals and beyond & GBSSI, Waxy & Metabolic enzyme & parallel & \cite{Jeon2010, Fan2008, Kawahigashi2013, Kawase2005, Hunt2012, Park2011}\\
 domestication & Grain quality &Maize & Species & species-specific, grass & Opaque2 & bZIP-TF & unknown & \citep{Martin2013}\\
 domestication & Shatter resistance &Sorghum, rice, maize & Family & grass-wide & Sh1, OsSh1, ZmSh1 & YABBY-like TF & parallel & \citep{Lin2012}\\
 both & Coloration &Pea, potato & Above family & outside the cereals & flavonoid 3',5'-hydroxylase & Metabolic enzyme & parallel & \citep{Martin2013}\\
 both & Coloration &Rice, potato & Species/above family & cereals and beyond & Rd/DFR, DFR & Metabolic enzyme & parallel & \citep{Furukawa2006, Zhang2009}\\
 both & Coloration &Blood orange & Species & species-specific, outside cereals & Ruby & MYB-TF & unknown & \citep{Butelli2012}\\
 both & Coloration &Rice & Species & species-specific, grass & Bh4 & Transporter protein & unknown & \citep{Zhu2011}\\
 both & Coloration &Soybean & Species & species-specific, outside cereals & R & MYB-TF & unknown & \citep{Gillman2011}\\
 both & Coloration &Rice & Species & species-specific, grass & Rc & bHLH-TF & unknown & \citep{Martin2013}\\
 both & Coloration &Grapevine & Species & species-specific, outside cereals & VvMYBA1-3 & MYB-TF & unknown & \citep{Martin2013}\\
 both & Flowering time &Barley, pea, strawberry & Above family & cereals and beyond & HvCEN, PsTFL1c, FvTFL1 & Signaling protein & parallel & \citep{Comadran2012, Foucher2003, Koskela2012}\\
 both & Flowering time &Rice, barley, pea, lentil & Family/above family & cereals and beyond & Hd17, EAM8, Mat-a, HR, LcELF3 & Circadian clock& parallel & \citep{Weller2012, Matsubara2012, Zakhrabekova2012, Faure2012}\\
 both & Flowering time &Rice, wheat, sunflower, barley & Family/above family & cereals and beyond & Hd3a (Heading date 3a), VRN3/TaFT, HaFT1, HvFT & Signaling protein & parallel & \citep{Yan2006, Takahashi2009, Blackman2010}\\
 both & Flowering time &Sorghum, rice & Family & grass-wide & Ghd7, SbGhd7 & CCT domain protein& parallel & \citep{Xue2008, Murphy2014}\\
 both & Flowering time &Turnip, Brassica oleracea & Family & outside the cereals & BrFLC2, BoFLC2 & MADS domain TF & parallel & \citep{Wu2012, Yuan2009, Okazaki2006}\\
 both & Flowering time &Barley, wheat, ryegrass & Species/family & grass-wide & VRN1, BM5, TmAP1, WAP1, LpVRN1 & MADS domain TF & parallel & \citep{Asp2011}\\
 both & Flowering time &Rice, barley, wheat, sorghum, sugar beet & Species/family/above family & cereals and beyond & OsPRR37, Ppd-H1, Ppd1, SbPRR37, BvBTC1 & Circadian clock& parallel & \citep{MURAKAMI2005, Turner2005, Jones2008, Beales2007, Wilhelm2008, Daz2012}\\
 both & Flowering time &Rice & Species & species-specific, grass & Hd1 & Zinc finger TF & unknown & \citep{Martin2013}\\
 both & Plant architecture &Maize, pearl millet, barley & Family & grass-wide & tb1, Pgtb1, INT-C & TCP-TF & parallel & \citep{Studer2011, Remigereau2011, Ramsay2011}\\
 both & Plant architecture &Barley & Species & species-specific, grass & VRS1 & Homeodomain-TF & unknown & \citep{Martin2013}\\
 adaptation & Cold tolerance &Barley, wheat & Family & grass-wide & HVA1, Wrab18, Wrab19 & LEA protein & parallel & \citep{Hong1988, pmid16755132}\\
 adaptation & Cold tolerance &Wheat, barley & Family & grass-wide & Wcs19, Wcor14, Wcor15, Bcor14b & Cor protein & parallel & \citep{Takumi2003}\\
 adaptation & Drought tolerance &Maize, Arabidopsis & Above family & cereals and beyond & ZmVPP1, AVP1 & H(+) pyrophosphatase & parallel & \citep{Wang2016}\\
 adaptation & Drought tolerance &Rice & Species & species-specific, grass & OsAHL1 & AT-hook PPC domain & unknown & \citep{Zhou2016}\\
 adaptation & Flowering time &Barley, wheat, maize & Species/family & grass-wide & VRN2, ZCCT1, ZmCCT9 & CCT domain protein& parallel & \citep{Huang2017}\\
 adaptation & Metal tolerance &Wheat, rye & Family & grass-wide & TaALMT1, ScALMT1 & Transporter protein & parallel & \citep{Martin2013}\\
 adaptation & Metal tolerance &Sorghum, Maize & Family & grass-wide & SbMATE1, ZmMATE1 & Transporter protein & parallel & \citep{Martin2013}\\
 adaptation & Pathogen resistance &Wheat, rice, sorghum & Family & grass-wide & LR34 & ABC transporter & parallel & \citep{Krattinger2010}\\
 adaptation & Pathogen resistance &Maize & Species & species-specific, grass & Rp3 & NBS-LRR & unknown & \citep{pmid12242248}\\
 adaptation & Soil salinity &Barley, maize, spinach & Above family & cereals and beyond & HvPIP2;1, ZmPIP2-4, PM28A & Aquaporin & parallel & \citep{Katsuhara2002, Zhu2005, Fotiadis2000}\\
 adaptation & Soil salinity &Rice, foxtail millet, tomato & Above family & cereals and beyond & OsASR1, OsASR3, SiASR1, SlASR1 & ABA stress ASR protein & parallel & \citep{Li2017, Konrad2008}\\
\end{tabular}
\end{center}
\end{sidewaystable}

\begin{figure}[h]
    \centering
    \includegraphics[width=15cm]{convergence_fig.pdf}
    \caption{Figure describing parallelism in convergence. Convergence is the phenomenon whereby similar traits, such as purple pigmentation in potato and rice in this example, arise independently in different species. Parallelism is when convergent traits are caused by modification of the same molecular pathways, genes, or nucleotides.
}
    \label{fig:convergence}
\end{figure}

\begin{figure}[h]
    \centering
    \includegraphics[width=15cm]{Figure_1.pdf}
    \caption{Simple cladogram of major cereal speciation. Numbers are in MYA (millions of years ago).
Orange sun: grass speciation event 75 MYA.  Blue stars: polyploidy events; 
the major grass polyploidy event immediately after the grass speciation event occurred 
approximately 70 MYA. The Ehrhartoideae clade, which includes rice, arose 
approximately 55MYA. The Pooideae clade, which includes wheat and barley, 
arose around 44MYA; Chloridoideae which contains foxtail millet 28 MYA, and the Panicoids, 
which include maize and sorghum, arose approximately 24MYA. The branch length is not 
proportional to the number of substitutions per site.
}
    \label{fig:grassphylo}
\end{figure}

\begin{figure}[h]
    \centering
    \includegraphics[width=15cm]{parallel_adaptation.pdf}
    \caption{Figure describing the likelihood that post-domestication adaptation would be parallel. The top panel describes a simple representation of spike domestication in wheat (top left) and maize (top right). Alongside the domestication of each crop is the resulting bottleneck. After the domestication of both crops, any adaptation must result from a population with lower diversity in each crop species. The likelihood of parallel adaptation is compromised due to this lower diversity, but parallelism may still result if certain criteria are met (bottom panel, on the left). 
}
    \label{fig:adaptation}
\end{figure}

\end{document}
