\title{Convergence of adaptation after domestication in the grasses}
\author{
        Woodhouse, M.R. and M. B. Hufford
}

\documentclass[12pt]{article}

\begin{document}
\maketitle

\begin{abstract}
The convergent evolution of desirable traits in crops during domestication has been well studied. In this review, the authors explore the current research to determine whether domestication bottlenecks in grass crops constrain environmental adaptation convergently, and how this phenomenon can help our understanding of the limitations in crop environmental adaptation.
\end{abstract}

\section{Introduction}
(NOTE: Matt, this draft is just my own interpretation of the topic. We do not need to use any of this if you think it is not good, or if you have other ideas on how to address this. I am a genome biologist more than a population biologist so I am writing from my area of expertise; it doesn't mean I think this is the only way to go! --MW)

There are two types of selection that domesticated species can undergo: the domestication process itself, whereby traits such as yield or early flowering are selected, and a later selection step whereby crops are selected to adapt to new conditions, such as elevation or drought. In the selection of favorable domestication traits, allelic diversity is often reduced in a population, causing a diversity bottleneck. The result is that if adaptation must subsequently occur, there is lower allelic diversity from which adaptation can arise. Domestication bottlenecks can thus impose constraints on what sort of adaptations are actually available in a population. For instance, if certain alleles in a population that give rise to drought tolerance are lost during domestication, adapting to drought conditions later on can only come about by 1) Mutation at the genes with the lost adaptive alleles; 2) Mutation at a different set of genes that can confer drought tolerance; or 3) Outcrossing to wild, more diverse relatives. A fourth scenario can be imagined: cases where alleles that contribute to desirable domestication traits also happen to be adaptive. How convergent any of these scenarios can be is the topic of this review.\paragraph{}

\section{Grasses as a single genetic system to study domestication}
It has been theorized that the grasses are closely related enough, yet divergent enough, to be viewed as a single genetic system (Bennetzen and Freeling 1993, Freeling 2001). This makes grasses especially conducive to a meaningful study of convergence. Grasses share certain features that permit similar types of selection, particularly dynamic genomes. All grasses share a genome duplication event 70-90 MYA 10.1073/pnas.0307901101, with maize undergoing a later ancient polyploidy event, and wheat having even more recent polyploids, including the tetraploid Triticum turgidum and the hexaploid T. aestivum, or modern bread wheat (refs needed). Grasses tend to  have relatively large genomes with active transposons, particularly maize and wheat (refs); they share certain domestication phenotypes, such as shatterproof seed, larger seed size, and early flowering (refs); and most grass crops were domesticated within the latitudinal boundaries of the equator and 35 N (Hawkes 1983; Harlan 1992; Smartt and Simmonds 1995), under similar environmental conditions, featuring both wet and dry seasons (Harlan 1992).  Nevertheless, each grass cereal has been cultivated separately in separate geographic locations, including maize (Americas), Sorghum (Africa), Rice (Asia), and Wheat and Barley (Middle East) (Note: not sure how relevant the last two points are, since I don't explore them at all later. However, if you have stuff to add that is germane to these topics, by all means, add to it! -MW). \paragraph{}

\section{Domestication bottlenecks in grass crops}
Domestication bottlenecks are often a result of selection for traits that make for desirable crops but not necessarily for environmental adaptability. Massive nucleotide diversity loss is reported in domesticated bread wheat (doi:10.1093/molbev/msm077), maize (with an increase in deleterious alleles) (PMID: 9539756, DOI 10.1186/s13059-017-1346-4), rice (PMID: 17218640), Sorghum (doi:  10.1534/genetics.105.054312), and barley (https://doi.org/10.1007/s00438-006-0136-6) compared with wild relatives, demonstrating that loss of diversity is widespread in cultivated grasses and is a phenomenon that is distinct from uncultivated wild relatives. Together these results suggest that domestication itself is responsible for the loss of diversity, and because of this, attempts to adapt domesticated grasses to new environments could pose a challenge. (This section might be a little thin. --MW)

\section{Post-domestication adaptation in grass crops: likelihood of convergence}
To what extent does diversity loss impact crop adaptation? Given how dynamic the grass genomes tend to be, as discussed, we would expect that even with a reduction in diversity, cereal cultivars could still adapt to environmental changes faster than less genomically dynamic species. The question is whether the initial loss of diversity always impacts the same genes that confer adaptation within these various cereal crops. Some lines of evidence would suggest this is not the case. Orphan genes, or genes that are specific to a particular lineage and share no syntenic orthologs with any outgroup, are known to code for traits involved in plant defense and stress response (http://dx.doi.org/10.1105/tpc.111; reviewed in https://doi.org/10.1016/j.tplants.2014.07.003). Orphan genes tend to be very dynamic, arising and becoming lost much faster than their syntenically retained, basal counterparts (http://www.genome.org/cgi/doi/10.1101/gr.081026.108). They do not tend to be involved in essential cellular or housekeeping processes, nor usually in development, which means they could be easily lost during domestication. If adaptation is dependent on these orphan-type genes, which quite often are unique even in individual cultivars within the same crop species, then we would not expect to see convergence at the alleleic level in cereal adaptation, since each species--indeed, each cultivar--would be expected to have its own unique, "outward-facing" suite of orphan genes that would confer environmental adaptation uniquely to its niche. Orphan genes often propagate through trans duplication (http://www.genome.org/cgi/doi/10.1101/gr.081026.108., https://doi.org/10.1371/journal.pgen.1000949); therefore, movement of these genes to a new region whose local euchromatic status can confer novel expression patterns to the mobilized gene can be a strong source of adaptation, especially since it has been shown that stressful environments can stimulate activation of transposable elements (https://doi.org/10.1104/pp.127.1.212; https://doi.org/10.1371/journal.pgen.1004915; reviewed in 10.3389/fpls.2016.01448), and perhaps the same or similar mechanisms can give rise to the mobilization of orphan genes. Taken together, these data suggest that environmental adaptation among different cereals would not be convergent, at least in regards to defense or stress response.

On the other hand, there are other traits thought to be involved in adaptation (?????Is what I say about these genes true, do you think? if you have better examples, that would be great! and we probably need some real-world examples of this kind of adaptation if it exists!-MW) that are often syntenically retained throughout the grasses, such as color genes (r/b, 10.1534/genetics.104.034629), vernalization (Vrn1, PMID: 12454082) and anoxic genes (Adh1/2, Clegg et al 199).  Some of these genes were likely originally part of the domestication process themselves, and so it stands to reason that they could possibly revert to an older, more adaptive phenotype if mutant alleles reverting it were selected for in a new environment favorable to it, which might happen more quickly in the grasses than in other types of crops; after all, the same genomic dynamism that can quickly evolve orphan genes can of course cause activated TEs themselves to transpose near genes that might confer novel expression patterns that are conducive to adaptation. The chances of a basal, syntenically retained domestication gene reverting to a more adaptive allele has therefore a higher likelihood to be convergent. However, if such "retro-adaptive" domestication genes existed as a single copy, then reversion to an adaptive phenotype could mean that the traits that made them desirable for domestication in the first place would be lost, and thus such alleles may not be selected for. But if such genes had post-polyploid homeologs, then their retained counterpart might become mutated instead so as to confer fitness without losing the desired domestication phenotype (Note: this is very hand-wave-y, I realize. But this is why we are writing this together, so you can tell me if I am full of it! :D -MW). This phenomenon of course would be reserved only for genomes such as maize or wheat that have had a more recent polyploid event. To some extent, however, it can be predicted which homeolog would be adaptive. It is known that of the two retained post-polyploidy subgenomes in maize, one undergoes less fractionation and is more highly expressed than the other (i.e. the dominant subgenome) (https://doi.org/10.1371/journal.pbio.1000409, https://doi.org/10.1073/pnas.1101368108). There is evidence that fractionation is biased not only in maize, but in wheat as well (10.1105/tpc.114.127183). Recent work has suggested that the genes on the more highly expressed subgenome in maize contribute more to phenotypic variation (doi:10.1093/molbev/msx121), and so it would be expected that selected adaptive alleles would likely be on the retained duplicate in the more dominant subgenome in both maize and wheat.  On the other hand, it is possible that subfunctionalization of both retained duplicates has already occurred, and that if the homeolog on the dominant subgenome is already conferring the domestication trait, plants with mutations on the homeolog on the nondominant subgenome might be more selected for adaptation. (Not really sure if genome dominance really adds much to the convergence argument, especially since the argument I use is very inconclusive. --MW)

\section{Conclusions}
The extent to which adaptation of cereal crops can be convergent is dependent on the environment the cultivar will be adapting to, since different adaptive phenotypes are dependent on either fast-evolving genes (stress- or defense response) or more basal genes (color, vernalization, other retro-adaptive domestication traits). Therefore, adaptation of cereal crops can be to some extent predictive, depending on the environment or the likelihood of adaptive alleles arising within certain loci within a population. 

\paragraph{Outline}
 Section~\ref{results}.
Finally, Section~\ref{conclusions} 



\bibliographystyle{abbrv}
\bibliography{simple}

\end{document}
This is never printed
